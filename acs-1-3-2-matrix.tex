\documentclass{beamer-control}
\usepackage{beamer-control-singlefile}
\INCLUDEONLY{The Matrix Exponential}
\begin{document}
\CONCEPT{The Matrix Exponential}

\begin{SUMMARY}
\begin{itemize}
\item Initial Condition Response
\item Diagonal and Jordan forms
\end{itemize}
\vfill References:
\begin{itemize}
\item \astrom{§6.2}
\end{itemize}
\end{SUMMARY}



\SUBCONCEPT{Initial Condition Response}

\begin{frame}{Generalising system output}
We have shown:
\begin{align}\label{eq:impresp}
y(t) = \int^t_0 \ImpulseResponse(t-\tau)u(\tau)\dee\tau
\end{align}
This form assumes zero initial conditions.

In this concept we:
\begin{itemize}
\item generalise  \eqref{impresp},
\item particularly for nonzero initial conditions.
\end{itemize}
\end{frame}

\begin{frame}
\frametitle{Consider $u=0$}
For unforced system:
\begin{align}
\dot x = A x
\end{align}
For scalar $A=a$, the solution is (naturally)
\begin{align}
x(t) = \ee^{at}x(0)
\end{align}
But note that $a$ could be complex!
\end{frame}

\begin{frame}
\frametitle{For multiple states}

For matrix $A$, we define solution
\begin{align}
x(t) = \ee^{At} x(0)
\end{align}
where $\ee^{At}$ is the \alert{matrix exponential}.
\bigskip

\begin{uncoverenv}<2>
Define $\ee^{At}$ using Taylor series:
\begin{align}
\ee^{At} = I + At + \tfrac12 A^2t^2 + \tfrac{1}{3!}A^3t^3 + \cdots = \sum_{k=0}^{\infty} \tfrac{1}{k!}A^kt^k
\end{align}
As expected: $\Deriv{}{t} \ee^{At} = A \ee^{At}$
\end{uncoverenv}
\end{frame}

\begin{frame}
\frametitle{Linearity of solutions}
Example two-state system:
\begin{align}
x(t) = \Matr{x_1\\x_2} = \ee^{At}\Matr{x_1(0)\\x_2(0)}
\end{align}
\begin{uncoverenv}<2->
Find solutions:
\begin{align}
x_{h1}(t) &= \ee^{At}\Matr{x_{1,0}\\0} &
x_{h2}(t) &= \ee^{At}\Matr{0\\x_{2,0}} 
\end{align}
\end{uncoverenv}
\begin{uncoverenv}<3->
For arbitrary initial condition:
\begin{align}
\ee^{At}\Matr{\alpha x_{1,0}\\\beta x_{2,0}} = \ee^{At}\Matr{\alpha x_{1,0}\\0} + \ee^{At}\Matr{0\\\beta x_{2,0}} = \alpha x_{h1}(t) + \beta x_{h2}(t)
\end{align}
\end{uncoverenv}
\end{frame}

\begin{frame}
\frametitle{Oscillator examples}
For undamped oscillator (mass-spring):
\begin{align}
\ddot q + \ww_0^2 q  = u
\end{align}
Dynamics are normalised using $x_1=q$, $x_2 = \dot q/\ww_0$:
\begin{align}
A &= \Matr{0 & \ww_0\\ -\ww_0 & 0} & \ee^{At} &= \Matr{ \cos \ww_0 t & \sin \ww_0 t \\ -\sin \ww_0 t & \cos \ww_0 t}
\end{align}
Therefore:
\begin{align}
x(t) = \ee^{At} x(0) = \Matr{ \cos \ww_0 t & \sin \ww_0 t \\ -\sin \ww_0 t & \cos \ww_0 t} \Matr{x_1(0)\\ x_2(0)}
\end{align}
\end{frame}

\begin{frame}
\frametitle{Oscillator examples}
For damped oscillator (mass-spring-damper):
\begin{align}
\ddot q + 2 \zz \ww_0 \dot q + \ww_0^2 q  = u
\end{align}
Skipping to the solution:
\begin{align}
x(t) = \ee^{At} x(0) = \ee^{-\zz\ww_0 t} \Matr{ \cos \ww_d t & \sin \ww_d t \\ -\sin \ww_d t & \cos \ww_d t} \Matr{x_1(0)\\ x_2(0)}
\end{align}
where $\wdamp = \ww_0 \sqrt{1-\zz^2}$ (damped natural frequency)
\end{frame}



\SUBCONCEPT{Diagonal and Jordan forms}

\begin{frame}
\frametitle{Diagonal forms}
In the undamped example, $A = \left[\begin{smallmatrix} 0 & \ww_0\\ -\ww_0 & 0\end{smallmatrix}\right]$ and we saw the solution was a linear combination of components.

\bigskip
For larger matrices, can generalise to \alert{diagonal form}:\footnote{\alert{This is the standard convention, opposite the textbook!}}
\begin{align}
\Deriv{x}{t} &= Ax & x &= Tz & \Deriv{z}{t} &= T^{-1}ATz = Jz
\end{align}
(Transformed states $z$ are non-physical)

\end{frame}

\begin{frame}[fragile]
\frametitle{Diagonal $A$ matrix}
Find in Matlab with:
\begin{lstlisting}[style=Matlab-editor]
[T,J] = jordan(A)
\end{lstlisting}
For $A = \left[\begin{smallmatrix} 0 & \ww_0\\ -\ww_0 & 0\end{smallmatrix}\right]$:
\begin{align}
T &= \Matr{ \ii & -\ii \\ 1 & 1  } & J &= T^{-1}AT = \Matr{ -\ww_0 \ii & 0 \\ 0 & \ww_0 \ii}
\end{align}
We can now see intuitively that the solutions will be of the form $\ee^{\ii \ww_0 t}$ which are sinusoids.
\end{frame}

\begin{frame}
\frametitle{Generalised diagonal matrix}
Recall solutions are in the form
\begin{align}
\ee^{At} = I + At + \tfrac12 A^2t^2 + \tfrac{1}{3!}A^3t^3 + \cdots = \sum_{k=0}^{\infty} \tfrac{1}{k!}A^kt^k
\end{align}
For diagonal $A$ matrices:
\begin{align}
A &= \Matr{ \lambda_1 &  &  & 0 \\  & \lambda_2 &  &  \\  &  & \ddots &  \\ 0 &  &  & \lambda_n } \,, &
(At)^k &= \Matr{ \lambda_1^kt^k &  &  & 0 \\  & \lambda_2^kt^k &  &  \\  &  & \ddots &  \\ 0 &  &  & \lambda_n^kt^k }
\end{align}
Combining:
\begin{align}\label{eq:diagform}
\ee^{At} = \Matr{ \ee^{\lambda_1 t} &  &  & 0 \\  & \ee^{\lambda_2 t} &  &  \\  &  & \ddots &  \\ 0 &  &  & \ee^{\lambda_n t} }
\end{align}

\end{frame}

\begin{frame}
\frametitle{Jordan form}
\begin{itemize}
\item Diagonal transformations can't get us to \eqref{diagform} if we have repeated eigenvalues ($\lambda_i=\lambda_j$)
\item In this case couple $k$ number of systems together:
\end{itemize}
\begin{align}
J   &= \Matr{ J_1 &  &  & 0 \\  & J_2 &  &  \\  &  & \ddots &  \\ 0 &  &  & J_k } \,, &
J_i &= \Matr{ \lambda_{i,1} & 1 &  & 0 \\  & \lambda_{i,2} & 1 &  \\  &  & \ddots &  \ddots \\ 0 &  &  & \lambda_{i,j} }
\end{align}
\end{frame}

\begin{frame}
\frametitle{Jordan form solution}
\begin{itemize}
\item Any square matrix can be written in Jordan form \AMref{Theorem 6.2}
\item Once in Jordan form, solution comes as:
\end{itemize}
\begin{align}
\ee^{At} = \Matr{ \ee^{J_1 t} &  &  & 0 \\  & \ee^{J_2 t} &  &  \\  &  & \ddots &  \\ 0 &  &  & \ee^{J_k t} }
\end{align}
with $\ee^{J_i t}$ given along subsequent lines \AMref{Eq (6.11)}
\end{frame}

\begin{frame}
\frametitle{The big reveal}
\begin{itemize}
\item All of this maths is to bring is to a fundamental truth which links linear algebra to ordinary differential equations
\item How do we look at an $A$ matrix to determine the properties of the system?
\item Transformations of $A$ bring us to Jordan form, where see all solutions in terms of $\lambda_i$ eigenvalues:
\end{itemize}
\begin{align}
\ee^{\lambda t} = \ee^{(\sigma + \ii \ww)t} = \ee^{\sigma t} ( \cos \ww t + \ii \sin \tt t)
\end{align}
Stability comes from $\sigma$, therefore:
\begin{theorem}
The system $\dot x=Ax$ is asymptotically stable if and only if all eigenvalues of $A$ have a strictly negative
real part and is unstable if any eigenvalue of $A$ has a strictly positive real part.
\end{theorem}
\end{frame}



\SUMMARYFRAME
\FINALE

\end{document}
