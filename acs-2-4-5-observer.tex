\documentclass{beamer-control}
\usepackage{beamer-control-singlefile}
\INCLUDEONLY{Output Feedback Basics}
\begin{document}
\CONCEPT{Output Feedback Basics}

\begin{SUMMARY}
\begin{itemize}
\item Observability 
\item State estimation
\item Control using estimated states
\end{itemize}
\vfill References:
\begin{itemize}
\item \astrom{§8.1–§8.3}
\end{itemize}
\end{SUMMARY}


\SUBCONCEPT{Observability}

\begin{frame}{Observability definition}
\begin{itemize}
\item In many settings, all states of a system cannot be measured and instead must be estimated using a model of the system
\item This can be done with a system known as an observer
\item For a system described by 
\[\frac{\mathrm{d}x}{\mathrm{d}t} = Ax+Bu, \quad y=Cx+Du\]
the system is observable if for every $T>0$, it is possible to determine the state of the system $x(T)$ through measurements of $y(t)$ and $u(t)$ for $0\leq t \leq T$
\item An observable system is therefore one whose dynamics can be understood just from measurement of the inputs and outputs
\end{itemize}
\end{frame}

\begin{frame}{Observability matrix}
	\begin{itemize}
		\item In a similar way to the reachability condition, for linear systems the \textit{observability} matrix tells us when it is possible to infer the state from observations of the output
		\item The observability matrix is defined as 
		\[W_o = \begin{bmatrix}
			C \\ CA \\ CA^2 \\ \vdots \\ CA^{n-1}
		\end{bmatrix}\]
		\item A linear system is observable if and only if the observability matrix $W_o$ is full row rank (it has $n$ independent rows)
	\end{itemize}
\end{frame}

\begin{frame}{Observability canonical form}
	\begin{itemize}
		\item We also may consider observability canonical form which is useful in studying observability
		\item Observable canonical form is given by 
		\small{\[\frac{\mathrm{d}z}{\mathrm{d}t} = \begin{bmatrix}
			-a_1 & 1 & & \\
			 -a_2 & 0 & \ddots & \\ \vdots & & \ddots & 1\\
			  -a_n & & & 0
		\end{bmatrix}z + \begin{bmatrix}
			b_1 \\ b_2 \\ \vdots \\ b_n
		\end{bmatrix} u \]
		\[y=\begin{bmatrix}
			1 & 0  & \cdots & 0
		\end{bmatrix}z + d_0u\]}
		\item The observability matrix in observability canonical form is 
		\small{\[\tilde{W}_o = \begin{bmatrix}
			1 & & & & \\
			a_1 & 1& & & \\
			a_2 & a_1 & 1 & & \\
			\vdots & \vdots & & \ddots & \\
			a_{n-1} & a_{n-2} & \cdots & a_1 & 1
		\end{bmatrix}^{-1}\]}
	\end{itemize}
\end{frame}

\SUBCONCEPT{State estimation}

\begin{frame}{The observer}
\begin{itemize}
\item Observers are linear dynamical systems that take the inputs $u$ and outputs $y$ of a system and produces an estimate $\hat{x}$ of the systems state $x$
\item The differential equations for an observer are
\[\frac{\mathrm{d}\hat{x}}{\mathrm{d}t} = A\hat{x}+Bu+L(y-C\hat{x})\]
\item The estimation error $\tilde{x}=x-\hat{x}$ therefore has dynamics
\[\frac{d\tilde{x}}{\mathrm{d}t} = (A-LC)\tilde{x}\]
so the error between our estimate and the true state goes to zero if $(A-LC)$ has eigenvalues with negative real part
\item The convergence rate is determined by appropriate selection of eigenvalues
\end{itemize}
\end{frame}

\begin{frame}{Eigenvalue assignment}
\begin{itemize}
	\item The problem of assigning eigenvalues for our observer is similar to the problem of assigning eigenvalues for a state feedback controller 
	\item Both problems are \textit{dual} with the following equivalences
	\[A \leftrightarrow A^T, \quad B \leftrightarrow C^T, \quad K \leftrightarrow L^T, \quad W_{\mathrm{r}} \leftrightarrow W_{\mathrm{o}}^T \]
	\item This means we can use the same algorithms to design  linear observers and linear controllers
\end{itemize}
\end{frame}

\begin{frame}{Observer design}
	\begin{enumerate}
		\item Given a linear state space model, $A$, $B$, $C$, with $D=0$, calculate the characteristic polynomial
		\[\operatorname{det}(sI-A) = s^n+a_1s^{n-1}+\cdots + a_{n-1}s+a_n\]
		\item If the system is observable then 
		\[\frac{\mathrm{d}\hat{x}}{\mathrm{d}t} = A\hat{x}+Bu+L(y-C\hat{x})\]
		is an observer for the system with $L$ chosen as 
		\[L= W_o^{-1} \tilde{W}_o\begin{bmatrix}
			p_1-a_1\\ p_2-a_2 \\ \vdots \\ p_n-a_n
		\end{bmatrix}\]
		
	\end{enumerate}
\end{frame}

\begin{frame}{Observer design}
	\begin{enumerate}
		\setcounter{enumi}{2}
		\item The observer error $\tilde{x}=x-\hat{x}$ is governed by the system with characteristic polynomial 
		\[p(s) = s^n+p_1 s^{n-1}+ \cdots p_{n-1}s+p_n\]
	\end{enumerate}
	This method is implemented in Matlab in the functions \textbf{acker} and \textbf{place} (same as for state feedback controller design!)
\end{frame}




\SUBCONCEPT{Control using estimated states}

\begin{frame}{Controllers and observers}
\begin{itemize}
	\item In settings where we do not have direct measurement of system states but we still wish to control our system, we may instead try feeding back our \textit{estimates} of the state
	\item Our control is then of the form 
	\[u=-K\hat{x}+k_f r\]
	where $\hat{x}$ is the output of the observer
	\[\frac{\mathrm{d}\hat{x}}{\mathrm{d}t} = A\hat{x}+Bu+L(y-C\hat{x})\]
	\item The full closed-loop system is then governed by the system dynamics and the observer error dynamics given by $\tilde{x}=x-\hat{x}$ as
	\[\frac{\mathrm{d}}{\mathrm{d}t} \begin{bmatrix}
		x \\ \tilde{x}
	\end{bmatrix} = \begin{bmatrix}
	A-BK & BK\\
	0 & A-LC 
	\end{bmatrix} \begin{bmatrix}
	x \\ \tilde{x}
	\end{bmatrix}  + \begin{bmatrix}
	Bk_f \\ 0
	\end{bmatrix} \]
\end{itemize}
	
\end{frame}

\begin{frame}{Controllers and observers}
\begin{itemize}
	\item Because the dynamics are block diagonal, the characteristic polynomial of the closed loop system is given by 
	\[p(s) = \operatorname{det}(sI-A+BK)\operatorname{det}(sI-A+LC)\]
	\item Therefore we must design the state feedback and observer simultaneously
	\item This involves a balance between high controller performance and robustness in the presence of uncertainty
\end{itemize}
\end{frame}


\SUMMARYFRAME
\FINALE

\end{document}
