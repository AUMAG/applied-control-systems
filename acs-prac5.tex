\documentclass[9pt]{beamer-control}
\usepackage{beamer-control-prac}
\begin{document}
\CONCEPT[5]{Week 5: Frequency domain modelling}

\begin{frame}
\frametitle{Introduction}
In this practical, ...

\vfill

This practical will consist of the following parts:
\begin{itemize}
\item ...
\end{itemize}
\end{frame}

\SUBCONCEPT{Second-order systems}

\begin{frame}{Compound pendulum}
	
Grab the yellow compound pendulum and position the QUBE on its side, such that the pendulum is appropriately affected by gravity.\\

\textcolor{red}{Image of compound pendulm and QUBE}

\end{frame}


\begin{frame}{Compound pendulum}
The nonlinear equation of motion for the pendulum is 
\[\tau(t) = (J_m+J_h+J_p) \ddot{\theta}(t) + D_r \dot{\theta}(t) + m_b r_p g \sin \left(\theta (t)\right), \]
where $\tau$ is the torque applied to the joint and $\theta$ is the angular displacement of the pendulum. The constants in this equation are the moments of inertia of the mass, hub, and pendulum $J_m$, $J_h$, and $J_p$, the mass of the bob $m_b$, centre of gravity of the pendulum $r_p$, viscous damping of the pivot $D_r$, and the gravitational acceleration $g$.\\

Using the small angle approximation, applying th Laplace transform, and rearranging the second-order transfer function for the compound pendulum is 
\[ G(s) = \frac{\theta(s)}{\tau (s)} = \frac{1}{(J_m+J_h+J_p)s^2 + D_r s + M_b r_p g } .\]

\end{frame}


\begin{frame}{Compound pendulum}
The constants for the QUBE compound pendulum plant have values XXXXXXX.
The moment of inertia of a pendulum is given by $J_p = m_b r_p^2+ \tfrac{1}{2}m_b r_b^2$.


\end{frame}

\begin{frame}{Compound pendulum}
The second-order transfer function may be transformed into a two-dimensional linear state-space model given by the matrices
\[\mathbf{A} = , \mathbf{B} = , \mathbf{C} = , \mathbf{D}=  \]
\end{frame}



\begin{frame}{Pole placement}

\textcolor{red}{Test this out with compound pendulum.}

Given the transfer function model of the compound pendulum, one method of control may be to design proportional feedback such that the resulting poles of the closed loop system provide a desired response. Using the MATLAB command K=place(A,B,p), determine the feedback gain required to move the poles such that the 2\% settling time is XXXX with a damping ratio of XXXX. \\
You may use the following relation
\[ T_{2\%} = \frac{4}{\zeta_n \omega_n}. \]

Why is pole placement not a robust control strategy?


\end{frame}


\begin{frame}{PID Control}
Now implement PID control, model response as a FOTD to find initial gains?
\end{frame}



\begin{frame}{Frequency response}
Given an input frequency, what is the phase and amplitude? \\

This will prepare us for the practical next week in which we will experimentally analyse the frequency response of the compound pendulum plant.
\end{frame}



\begin{frame}{Next week}
	This week you .....
	
	Introductory sentence...... Next week ..... 
\end{frame}



\end{document}
