\documentclass[9pt]{beamer-control}
\usepackage{beamer-control-prac}
\begin{document}
\CONCEPT[6]{Week 6: Frequency domain analysis}

\begin{frame}
\frametitle{Introduction}
In this practical, ...

\vfill

This practical will consist of the following parts:
\begin{itemize}
\item ...
\end{itemize}
\end{frame}

\SUBCONCEPT{Bode, Nichols, and Nyquist}

\begin{frame}{Bode}
\begin{itemize}
\item a
\item b
\end{itemize}
\textcolor{red}{Use figures from slide notes}
\end{frame}

\begin{frame}{Nichols}
	\begin{itemize}
		\item a
		\item b
	\end{itemize}
\textcolor{red}{Use figures from slide notes}
\end{frame}

\begin{frame}{Nyquist}
	\begin{itemize}
		\item a
		\item b
	\end{itemize}
\textcolor{red}{Use figures from slide notes}
\end{frame}

\SUBCONCEPT{Theoeretical analysis}

\begin{frame}{Frequency analysis}
Based on the given model of the compound pendulum, use MATLAB to plot the Bode, Nichols, and Nyquist curves.
\end{frame}

\SUBCONCEPT{Experimental analysis}

\begin{frame}{Frequency analysis}
The Bode, Nichols, and Nyquist curves can all be constructed from knowledge of the phase and amplitude of the output signal of a plant driven at a range of input frequencies

\end{frame}


\begin{frame}{Next week}
	This week you .....
	
	Introductory sentence...... Next week ..... 
\end{frame}



\end{document}
