\documentclass[9pt]{beamer-control}
\usepackage{beamer-control-workshop}
\begin{document}
\CONCEPT[2]{Week 2: Dynamic Behaviour}

\begin{frame}
\frametitle{Introduction}
In this workshop, we will use the examples from the concepts notes to explore properties of nonlinear systems.
\end{frame}

\SUBCONCEPT{Vector field and phase portrait}

\begin{frame}
\frametitle{A nonlinear ODE}

Satisfy yourself that you can create a nonlinear mass spring damper model:

\includematlab{concept2_2.m}{msdnl}

Double-check you can see what the nonlinearity is here. (This is not necessarily physical, it's just supposed to be a bit out of the ordinary compared to a linear mass-spring-damper.)

\end{frame}

\begin{frame}
\frametitle{Plotting state variables vs time}
We simulate and plot as usual:

\includematlab{concept2_2.m}{plot1}

Now add another dynamics function which is linear with same code as before but with sum of forces line replaced by:

\includematlab{concept2_2.m}{msdlin}

Compare the time responses of the two and check they look qualitatively different. (The factor of 100 difference is just because the nonlinearity makes the coefficient have a different scale of effect.)

\end{frame}

\begin{frame}
\frametitle{Vector field}

\begin{itemize}
\item
To plot the vector field, evaluate $\dot x$ at a grid of points $(x_{1,i},y_{2,j})$.

\item
We have the calculation via \texttt{mass\_spring\_damper\_nonlin}, and the way we have written it is automatically `vectorised' --- this means we can pass it a vector of inputs and it will evaluate the function on each element of the vector then return them all in an output vector.

\item
Matlab provides a suite of tools to help vectorise code. Here we define the range of $x_1$ and $x_2$ we want via \texttt{linspace}. The \texttt{meshgrid} command then gives us an array of every combination of $x_1$ and $x_2$.

\item
Finally, the \texttt{transpose} line puts $x_1$ and $x_2$ into a horizontal stacked series of column vectors, like:
\[
   \Matr{\Matr{x_{1,1}\\x_{2,1}}\Matr{x_{1,2}\\x_{2,2}}\Matr{x_{1,3}\\x_{2,3}}}
\]
\end{itemize}

\includematlab{concept2_2.m}{quiv1}

\end{frame}

\begin{frame}
\frametitle{Vector field plotting}

\begin{itemize}
\item
Calculating the vector field is the first step, now we plot
\item
Matlab provides \texttt{quiver} to do this, it takes the following arguments:
\begin{itemize}
\item Array of $x_1$ points
\item Array of $x_2$ points
\item Array of $\dot x_1$ values
\item Array of $\dot x_2$ values
\end{itemize}
\item As long as the array sizes are consistent, Matlab will plot the vectors at each point
\item Once you've done that for the nonlinear system, compare with the linear one --- what do you see?
\end{itemize}

\includematlab{concept2_2.m}{quiv2}

\end{frame}

\begin{frame}
\frametitle{Phase portraits}
\begin{itemize}
\item
Phase portraits are arguably conceptually easier, but need a for loop so look a bit more complicated
\item
All you need to do is pick a bunch of initial conditions, simulate each scenario, and instead of plotting the response vs time, plot $x_2$ vs $x_1$
\item
This code plots both $x$ vs $t$ side-by-side with $x_2$ vs $x_1$  so you can explicitly compare them
\item
Labels and axis scaling omitted for brevity
\item
Once again compare to the linear system and see what you think
\end{itemize}

\includematlab{concept2_2.m}{phase}

\end{frame}

\SUBCONCEPT{Cruise control example}

\begin{frame}
\frametitle{Cruise control example}
\begin{itemize}
\item Download the provided code for the cruise control example in Concept~2.4
\item Work through the file block by block --- you can run each sequentially using CTRL+ENTER (Windows) or CMD+RETURN (Mac)
\item Confirm that the relationship between the mathematics and the code is clear in your head
\begin{itemize}
\item We don't expect you to be able to sit down and write this from scratch, yet, but perhaps later, yes?
\end{itemize}
\item Now examine the PI controller
\begin{itemize}
\item
Later in this course, quite some focus will be on ``what happens when you change control gains?''
\item
What happens as the $k_p$ gain is made higher or lower? Note how the performance changes
\end{itemize}
\end{itemize}
\end{frame}


\end{document}
