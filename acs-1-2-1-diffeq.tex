\documentclass{beamer-control}
\usepackage{beamer-control-singlefile}
\INCLUDEONLY{Solving Differential Equations}
\begin{document}
\CONCEPT{Solving Differential Equations}

\begin{SUMMARY}
\begin{itemize}
\item A little more formal treatment of ODEs
\item Different solution types 
\end{itemize}
\vfill References:
\begin{itemize}
\item \astrom{§5.1}
\end{itemize}
\end{SUMMARY}



\begin{frame}{Recalling our differential equations}
Recall our ODE:
\begin{align}
\EqStd
\end{align}
Let's be more formal:
\begin{align}
x &= \Matr{x_1\\x_2\\\vdots\\x_n} \in \Reals^n & 
u &= \Matr{u_1\\u_2\\\vdots\\u_p} \in \Reals^p & 
y &= \Matr{y_1\\y_2\\\vdots\\y_q} \in \Reals^q 
\end{align}
Smooth maps:
\begin{align}
f&\colon\Reals^n \times \Reals^p \to \Reals^n &
h&\colon\Reals^n \times \Reals^p \to \Reals^q
\end{align}
\end{frame}

\begin{frame}
\frametitle{Now add control}
When we design a controller, the input $u$ is generally determined by the state:\footnote{Note: I don't like using uppercase here\dots}
\begin{align}
u &= \alpha(x) & \Deriv{x}{t} &= f(x,u) = f(x,\alpha(x)) \coloneq F(x) \label{eq:Fx}
\end{align}
At this point, $\alpha\colon\Reals^n \to \Reals^p$ could be any smooth nonlinear function 
\bigskip

If we can solve $\dot x=F(x)$ we know how the system behaves---what types of solution are possible?
\end{frame}

\begin{frame}
\frametitle{Solution types}
\begin{itemize}
\item We are searching for a specific $x(t)$ which satisfies $\dot x(t)=F(x(t))$
\item Usually care about \emph{initial value problems}
\item Usually there is a unique solution for the particular system
\item Usually there are sinusoids and exponential functions :)
\end{itemize}
\end{frame}

\begin{frame}
\frametitle{Example: damped oscillator}
\begin{itemize}
\item This linear system has a unique analytical solution
\item Note that the solution depends on the damping ratio ($0<\zz<1$)
\item D.E.s with non-unique solutions tend to be nonphysical
\end{itemize}
\end{frame}

\begin{frame}
\frametitle{Example: finite escape time}
\begin{align}
\dot x &= x^2 & x(t) &= \frac{1}{1-t}
\end{align}
\begin{itemize}
\item What happens as $t\to1$ ?
\item $x(t)$ has finite escape time
\end{itemize}
\end{frame}

\begin{frame}
\frametitle{Another example (not in the textbook)}
\begin{align}
\dot x &= x \\
x(t)   &= \ee^{t} \\
x(0)   &= \\
x(t)|_{t\to\infty}&=
\end{align}

\vfill
\hrulefill
\vfill

\begin{align}
\dot x &= -x \\
x(t)   &=  \\
x(0)   &= \\
x(t)|_{t\to\infty}&=
\end{align}
\end{frame}


\SUMMARYFRAME
\FINALE

\end{document}
